\documentclass[letterpaper,11pt]{article}

\usepackage{amsfonts}
\usepackage{graphicx}
\usepackage[left=2cm,top=2cm,right=2cm,bottom=2.5cm]{geometry}
\usepackage{url}
\usepackage{subfig}
\usepackage{float}
\usepackage{setspace}
\usepackage{lineno}
\usepackage{natbib}
\usepackage{amsmath}
\usepackage{xr}

\newcommand{\hA}{\hat{A}}
\newcommand{\hR}{\hat{R}}
\newcommand{\hM}{\hat{M}}
\newcommand{\hN}{\hat{N}}
\newcommand{\halph}{\hat{\alpha}}
\newcommand{\bmax}{b_{\text{max}}}
\newcommand{\fmin}{f_{\text{min}}}
\newcommand{\fmax}{f_{\text{max}}}

\title{Appendix for ``Symbiont coexistence may be common in mutualisms because of negative physiological feedback in preferential allocation and resource partitioning''}

\author{Shyamolina Ghosh, \and Daniel C. Reuman, \and James D. Bever}
\date{}

\begin{document}

\maketitle

\section{Assumptions on model parameter values}\label{sect:assumptions}

\noindent We adopt the following biologically reasonable assumptions:
\begin{enumerate}
\item $0<f<1$;
\item $a_N \geq a_M > 0$;
\item $s>0$;
\item $\bmax-d>0$;
\item $\bmax(1-s)-d>0$;
\item $0<P_S<1$;
\item $D>0$;
\item $0<e<1$;
\item $0<u<1$;
\item $K_A>0$;
\item $K>0$.
\end{enumerate}
These assumptions are referred to by number in the derivations that follow.
%***DAN: make sure in the development below you always cite an assumption when you make one. So far you do not always do that.

\section{Model equilibrium}\label{sect:equil}

\noindent A positive equilibrium of the dynamical model, here denoted $\hA>0,\hR>0,\hM>0,\hN>0$, occurs if and only if
a positive solution exists to the the system of equations obtained by setting the right sides of (1),(3),(4),(6)
%***DAN: note that references to the main text are not dynamic links
in the main text to $0$. So we examine that system of equations. 
Setting $\frac{dM}{dt}=0$ and $\frac{dN}{dt}=0$, substituting in the expressions for $C_M$ and $C_N$, and carrying out
algebraic manipulations leads to
\begin{align}
\frac{\hA}{\halph} &= -ea_M \hR +\frac{Kd}{\bmax (1-s)-d} \\
\frac{\hA}{\halph} &= \frac{-ea_N}{1-f}\hR +\frac{Kd}{(\bmax-d)(1-f)},
\end{align}
where
\begin{equation}
\halph=(\hM+\hN)\left[1-f+f\frac{\hM}{\hM+\hN} \right].
\end{equation}
Therefore we consider the two lines in $xy$-space given by
\begin{align}
y &= -ea_M x +\frac{Kd}{\bmax (1-s)-d} \label{Mline}\\
y &= \frac{-ea_N}{1-f}x +\frac{Kd}{(\bmax-d)(1-f)}. \label{Nline}
\end{align}
The $x$-intercepts of these lines are $\frac{Kd}{ea_M(\bmax(1-s)-d)}$ and
$\frac{Kd}{ea_N(\bmax-d)}$, respectively. By assumption 2 listed above, $\frac{a_N}{a_M}\geq1$, and by assumptions 3-5,
$\frac{\bmax(1-s)-d}{\bmax-d}<1$. So therefore $\frac{a_N}{a_M}>\frac{\bmax(1-s)-d}{\bmax-d}$, and
therefore $\frac{Kd}{ea_M(\bmax(1-s)-d)}>\frac{Kd}{ea_N(\bmax-d)}$, i.e. the $x$-intercept of (\ref{Mline})
is strictly to the right of the $x$-intercept of (\ref{Nline}). Likewise, by assumption 1 and 2, the line given by
(\ref{Nline}) is steeper (has a more-negative slope) than the line give by (\ref{Mline}). So the two lines have a
positive intersection point if and only if the $y$-intercept of (\ref{Nline}) is strictly greater than the
$y$-intercept of (\ref{Mline}), i.e.,
\begin{equation}
\frac{Kd}{\bmax(1-s)-d}<\frac{Kd}{(\bmax-d)(1-f)}.
\end{equation}
After some algebra, this becomes
\begin{equation}
f>\frac{\bmax s}{\bmax-d}.\label{FirstNecessaryCondition}
\end{equation}
The condition (\ref{FirstNecessaryCondition}) is our first necessary condition for the existence of a positive equilibrium to
the model dynamical equations ((1),(3),(4),(6) in the main text). We denote $\fmin=\frac{\bmax s}{\bmax-d}$. By assumption 5,
$\bmax(1-s)-d>0$, so $d<\bmax(1-s)$, so $d<\bmax-\bmax s$, so $\bmax s < \bmax - d$, so $\fmin<1$. Thus no additional assumptions
are needed to conclude that (\ref{FirstNecessaryCondition}) can be satisfied for some $0<f<1$.

If the condition (\ref{FirstNecessaryCondition}) is satisfied, then there is a positive solution to (\ref{Mline})-(\ref{Nline}),
giving us (after some algebra) the positive quantities
\begin{align}
\hR &= \frac{Kd(1-f)}{e(a_N-a_M (1-f))}\left[ \frac{1}{(\bmax-d)(1-f)}-\frac{1}{\bmax(1-s)-d} \right] \\
&= \frac{Kd(f-\fmin)}{e(a_N-a_M(1-f))(\bmax-d)(1-\fmin)} \label{SecondhRExpression}\\
\frac{\hA}{\halph} &= \frac{Kd(\bmax-d)(a_N-a_M)+sKd\bmax a_M}{(\bmax-d)(\bmax(1-s)-d)(a_N-a_M(1-f))}.
\end{align}
We can also see directly that these are positive, using the assumptions listed above.

Now, setting $\frac{dA}{dt}=0$, we get
\begin{align}
0 &= 1-P_S -\hA F(\hM,\hN) \\
\hA &= \frac{1-P_S}{F(\hN,\hM)} \\
&= \frac{(1-P_S)\halph}{u\left( \frac{\hM}{K_A+\hM} \right)\hM} \\
&= \frac{(1-P_S)\halph (K_A+\hM)}{u \hM^2},
\end{align}
which leads to the quadratic equation
\begin{equation}
u\frac{\hA}{\halph}\hM^2-(1-P_S)\hM-(1-P_S)K_A=0.
\end{equation}
The solutions to this quadratic equation are
\begin{equation}
\hM = \frac{(1-P_S) \pm \sqrt{(1-P_S)^2+4u\frac{\hA}{\halph}(1-P_S)K_A}}{2u\frac{\hA}{\halph}}.
\end{equation}
But, by assumptions 6, 9 and 10, 
\begin{equation}
4u\frac{\hA}{\halph}(1-P_S)K_A>0, 
\end{equation}
so 
\begin{equation}
(1-P_S)^2+4u\frac{\hA}{\halph}(1-P_S)K_A>(1-P_S)^2, 
\end{equation}
so
\begin{equation}
\sqrt{(1-P_S)^2+4u\frac{\hA}{\halph}(1-P_S)K_A}>1-P_S,
\end{equation}
so the solution
\begin{equation}
\hM = \frac{(1-P_S) - \sqrt{(1-P_S)^2+4u\frac{\hA}{\halph}(1-P_S)K_A}}{2u\frac{\hA}{\halph}}
\end{equation}
is negative. Since we are looking for positive solutions, we ignore this one, considering only
\begin{equation}
\hM = \frac{(1-P_S) + \sqrt{(1-P_S)^2+4u\frac{\hA}{\halph}(1-P_S)K_A}}{2u\frac{\hA}{\halph}},
\end{equation}
which is positive, by the assumptions.

Now setting $\frac{dR}{dt}=0$, we get
\begin{equation}
0 = D-(a_N \hM + a_N \hN)\hR.
\end{equation}
Solving for $\hN$ gives
\begin{equation}
\hN = \frac{D}{a_N \hR} -\frac{a_M}{a_N}\hM.
\end{equation}
This is positive whenever
\begin{equation}
\hM \hR < \frac{D}{a_M}.\label{FinalCondition}
\end{equation}
In the limit as $f \rightarrow \fmin$ from the right, $\hR \rightarrow 0$ from above, so (\ref{FinalCondition}) is satisfied
for $f$ sufficiently close to $\fmin$. The derivative of $\hR$ with respect to $f$, computed
starting from (\ref{SecondhRExpression}), is
\begin{equation}
\frac{d\hR}{df}=\frac{Kd}{e(\bmax -d)(1-\fmin)}\frac{\fmin a_M +a_N-a_M}{(f a_M +a_N-a_M)^2},
\end{equation}
which is positive, by the assumptions. Therefore $\hR$ increases as $f$ increases from $\fmin$.
Likewise, setting
\begin{align}
\eta &= \frac{\halph}{\hA} \\
&= \frac{a_M (\bmax - d)(\bmax (1-s)-d) f +(a_N-a_M)(\bmax-d)(\bmax(1-s)-d)}{Kd(\bmax-d)(a_N-a_M)+sKd\bmax a_M},
\end{align}
we can write
\begin{equation}
\hM = \frac{1-P_S}{2u}\eta + \frac{1}{2u}\sqrt{(1-P_s)^2 \eta^2 + 4u (1-P_S)K_A \eta}.\label{Mveta}
\end{equation}
Again making use of the assumptions listed above, it is straightforward to see that $\eta$ is a
linear, positive-slope function of $f$, and therefore increases as $f$ increases. But (\ref{Mveta}) indicates
that $\hM$ increases as $\eta$ increases, so $\hM$ increases as $f$ increases from $\fmin$. Thus the
condition (\ref{FinalCondition}) is satisfied for $f$ up to some value, and then is not satisfied for $f$ equal
to that value or higher than that value. We take $\fmax$ to be the minimum
of this value and $1$. Although we have not computed a closed-form solution for $\fmax$, for any specific values
of model parameters it would be straightforward and computationally fast to compute $\fmax$ to any desired precision
by computing $\hM$ and $\hR$ for a range of values of $f$ increasing from $\fmin$, and then evaluating for which
values the condition (\ref{FinalCondition}) is satisfied.

In summary, the calculations here indicate that $\fmin<f<\fmax$ is a necessary and sufficient condition
for the existence of a positive equilibrium of the model dynamical equations ((1), (3), (4), (6) in the main text).
The equilibrium is then given by
\begin{align}
\hR &= \frac{Kd(f-\fmin)}{e(a_N-a_M(1-f))(\bmax-d)(1-\fmin)} \\
\frac{\hA}{\halph} &= \frac{Kd(\bmax-d)(a_N-a_M)+sKd\bmax a_M}{(\bmax-d)(\bmax(1-s)-d)(a_N-a_M(1-f))} \\
\hM &= \frac{(1-P_S) + \sqrt{(1-P_S)^2+4u\frac{\hA}{\halph}(1-P_S)K_A}}{2u\frac{\hA}{\halph}} \\
\hN &= \frac{D}{a_N \hR}-\frac{a_M}{a_N}\hM \\
\hA &= \frac{\hA}{\halph}(\hM+(1-f)\hN).
\end{align}
For any given values of the parameters (1), (3), (4), (6) in the main text, the model equilibrium can be obtained
straightforwardly by evaluating these expressions in turn.

\section{Stability of the model equilibrium}\label{sect:stability}

\noindent We evaluated the local asymptotic stability of the model equilibrium numerically, as follows. 
We selected model parameters within the ranges specified by Table \ref{tab:paramranges}, independently
and following uniform distributions across the specified ranges for each parameter. For each set of parameters that 
satisfied the assumptions listed above (section \ref{sect:assumptions}) we computed the range of
$f$ values $\fmin$ to $\fmax$. The bound $\fmax$ was estimated with a root-finding algorithm in
the R programming language (the function `uniroot' in the `stats' package) which had very good but finite 
precision. If the imprecision for $\fmax$ reported by uniroot was more than $1/50$th the distance 
between $\fmin$ and $\fmax$, parameters were not used. Otherwise, $27$ values of $f$ were spaced 
evenly from $\fmin$ and $\fmax$, the first and last were discarded (these were $\fmin$ and $\fmax$
themselves), and the other $f$ values were combined with values for the other parameters. For these parameter sets, 
the model Jacobian was calculated at the equilibrium, eigenvalues of the Jacobian were
computed using the `eigen' function in R, and the maximal real part of these eigenvalues was retained. 
If this real part was negative, the model equilibrium for those parameters was stable. Stability was checked
for 297925 parameter combinations, and in every case the model equilibrium was stable. Although this does not 
guarantee that the model equilibrium we described (section \ref{sect:equil}) is always stable, it does 
indicate that stability is common for reasonable model parameters.

\begin{table}
\begin{center}
\caption{Ranges used for model parameters in evaluating stability of the equilibrium. See section \ref{sect:stability}.}
\label{tab:paramranges}
\begin{tabular}{ccc}
\hline
\textbf{Parameter} & \textbf{Lower} & \textbf{Upper} \\
\hline
$u$ & 0.001 & 1 \\
$\bmax$ & 0.001 & 5 \\
$d$ & 0.001 & 5 \\
$s$ & 0.001 & 0.999 \\
$e$ & 0.001 & 0.999 \\
$a_M$ & 0.001 & 5 \\
$a_N$ & 0.001 & 5 \\
$D$ & 0.001 & 5 \\
$K_A$ & 0.01 & 15 \\
$K$ & 0.01 & 15 \\
$P_S$ & 0.001 & 0.999 \\ 
\hline
\end{tabular}
\end{center}
\end{table}

\section{Alternative form of the efficiency of phosphorus return in Eq. 1}\label{sect:altmodel}

Here we start by rewriting the functional form of phosphorus return from Eq. (1) of main text as follows:

\begin{align}
F(M,N) &= u \left( \dfrac{M}{M+K_{A}}\right) \left\lbrace  \dfrac{M/(M+N)}{1-f+f\lbrace M/(M+N)\rbrace }\right\rbrace\\
or,\;\ F(M,N) &= F_{1}(M,N) = u \left( \dfrac{M}{M+K_{A}}\right) \left\lbrace \dfrac{M}{M+N(1-f)} \right\rbrace   \label{fmn1}
\end{align}


One can consider a simpler and alternative form for $F(M,N)$ in Eq. (1) as follows;
\begin{align}
F(M,N) &= u \left( \dfrac{M}{M+K_{A}}\right) \left\lbrace f+(1-f)\left( \dfrac{M}{M+N}\right) \right\rbrace \\
or,\;\ F(M,N) &= F_{2}(M,N) = u \left( \dfrac{M}{M+K_{A}}\right) \left\lbrace \dfrac{M+Nf}{M+N} \right\rbrace  \label{fmn2}
\end{align}
$F(M,N)$ from Eq. \ref{fmn2} is depicted in Fig. \ref{fig:fmn2} and its shape is very similar to Fig. 1(a) in the main text.

We simulated Eqs. (1) - (7) with $F_{2}(M,N)$ as $F(M,N)$ in Eq. (2) and got qualitatively same results as of Fig. 2 in the main text. This happened because the two terms $F_{1}$ and $F_{2}$ are same at equilibrium (see Table 2) for $f \rightarrow 0$, $f \rightarrow f_{min}$, $f \rightarrow f_{max}$ and $f \rightarrow 1$. Therefore, we can say that our finding is robust in nature.

\begin{table}
\begin{center}
\caption{Comparing two functional forms at the equilibrium. See section \ref{sect:altmodel}.}
\label{tab:compareFMN}
\begin{tabular}{ccccc}
\hline
$F(M,N)$ at eqm. & when $f \rightarrow 0$ & when $f \rightarrow f_{min}$ & when $f \rightarrow f_{max}$ & when $f \rightarrow 1$\\
\hline
$F_{1}(\hM,\hN) \rightarrow$ & $u \left( \dfrac{\hM}{\hM+K_{A}}\right) \left(\dfrac{\hM}{\hM+\hN}\right)$ & 0 & $u \left( \dfrac{\hM}{\hM+K_{A}}\right)$ & $u \left( \dfrac{\hM}{\hM+K_{A}}\right)$ \\
$F_{2}(\hM,\hN) \rightarrow$ & $u \left( \dfrac{\hM}{\hM+K_{A}}\right) \left(\dfrac{\hM}{\hM+\hN}\right)$ & 0 & $u \left( \dfrac{\hM}{\hM+K_{A}}\right)$ & $u \left( \dfrac{\hM}{\hM+K_{A}}\right)$\\
\hline
\end{tabular}
\end{center}
\end{table}

\begin{figure}
\includegraphics[scale=0.5]{./ARMN_Results/Puptake_vs_M_N_alternative.pdf}
\caption{P-uptake via mycorrhizal fungi, $F_2(M,N)$ as a saturating function of densities of mutualists
$(M)$ and non-mutualists $(N)$ (Eq. \ref{fmn2}). Parameters used: $f = 0.3, u = 0.4, K_A = 5$.}\label{fig:fmn2}
\end{figure}


\end{document}

